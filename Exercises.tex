% The header contains the main properties of the article.
\documentclass[12pt,a4paper]{article} % 12 Punkte Schrift, A4 Papier
\usepackage[utf8]{inputenc} 
%\usepackage[ngerman]{babel} % Use if you want to write in German. 
\usepackage[english]{babel} % Use if you want to write in English

\parindent=0cm
\parskip=0.3cm
\linespread{1.5}




% some standard packages.
\usepackage{amsmath}
\usepackage{amsfonts}
\usepackage{amssymb}
\usepackage{pdfpages}
\usepackage{hyperref}
\usepackage{blindtext}
\usepackage{mhchem}
\usepackage{chemfig}




% Information for the title page
\title{Example for a Report}
\date{\today}
\author{Suzy Sample}




% The Beginning of the document
\begin{document}


\section*{Exercises}
\begin{enumerate}
\item Copy all files in the 'prep' folder to your personal documents folder.
\item Open the file "Report\_Template.tex" with texmaker and compile twice (Quick Build).
\item Add new sections (\verb|\section{...}|), subsections (\verb|\subsection{...}|) and subsubsections (\verb|\subsubsection{...}|). 
\item Add a new section with a star (\verb|\section*{...}|). What does the star do?
\item 	In the section on mathematical forumulae add the formula for the area of a circle, the cosine rule 
	\begin{align*}
		c^2 &= a^2 + b^2 - 2 \cdot  a \cdot  b \cdot  \cos (\gamma)
	\end{align*}
	and Newton's Law of Gravity
	\begin{align*}
		F &= \Gamma \cdot  \frac{\mu_1 \cdot  \mu_2}{\rho^2}
	\end{align*}
	inline  (\texttt{\$\dots\$}) and in a separate ennvironment (\texttt{\textbackslash[\ldots\textbackslash]}).
	\item Add more colorful tables to the section on tables.
	\item Add more images in the section on figures. Use \texttt{width=} and \texttt{height=} to change the size of your images.
	\item Add a footnote.
	\item Add a citation.
	\item Add an item to the bibliography.
	\item Add the chemical formula
		\begin{center}
		\ce{SO3 + H2SO4 <=> H2S2O7 <=> HSO3+ + HSO4-}
	\end{center}
	\item Add the chemfig package in the preamble then write the structure formula for propane to the chemical formula section:
		\begin{center}
		\chemfig{H-C(-[2]H)(-[6]H)-C(-[2]H)(-[6]H)-C(-[2]H)(-[6]H)-H}
	\end{center}
\end{enumerate}




\end{document}